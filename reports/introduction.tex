\section{Introduction}

The Cox proportional hazards model \cite{cox} is a core component of several experimental designs in medicine and beyond. These experimental results impact the lives of millions of patients globally and billions of dollars in investment. Little is understood, however, regarding which design elements most contribute to a design's success or failure; i.e. the study's statistical power. Scientists can design more effective and cost-effective studies by understanding which design elements impact statistical power.

\subsection{An Illustrative Example}

By way of example, consider the hypothetical osteopathic doctor, Caitlin. Caitlin runs a private practice and wants to verify that ibuprofen helps to break a fever. Flu season starts and she accepts 10 feverish patients daily. Each consenting patient is prescribed ibuprofen and reports their body temperature back to Caitlin in the evening. 

After the first day, Caitlin understands that she probably doesn't have enough data to find an effect; her data lacks statistical power. Almost every patient is still feverish because not enough time has passed for the ibuprofen to break the fever. 

On the 10\textsuperscript{th} day, Caitlin thinks she has enough data to test her hypothesis. She wonders, however, if she is correct to remove the 10\textsuperscript{th} day's data from her analysis. For the same reason she did not model the first day's patients for lack of statistical power, so too should she omit the patients on the 10\textsuperscript{th} day? And if the 10\textsuperscript{th} day is invalid, does she remove days nine and eight? She doesn't know the answer, but she's confident she should filter out day 10 at a minimum. 

One solution is to stop introducing new patients to the study and to analyze the results after every existing patient reports that their fever is gone. This ensures an uncensored dataset and delays her study indefinitely. If ibuprofen is a fever reducer, then Caitlin misses the opportunity to treat new patients more effectively. While the cost of waiting is small when treating a fever, a more aggressive ailment is more costly. 

A second solution is to observe the data and pick the cut-off point that looks best, but Caitlin is wary of biasing the dataset by choosing data she determines valuable. Her results may be called into question and compromise her integrity as a medical professional. 

How then, can Caitlin honestly and ethically subset her dataset to maximize the power of the Cox proportional hazards model? The answer is important, not only to Caitlin, but to medicine in general. If we can design a better study, then we can test results and apply the resulting treatments to patients faster. 

\subsection{Existing Literature}

There is some existing literature on the topic of power analysis for the Cox proportional hazards model \cite{bender, gonen, hsieh}, but none recommend how to filter a dataset to maximize statistical power of the Cox proportional hazards model. This is a novel approach as far as we can tell.
