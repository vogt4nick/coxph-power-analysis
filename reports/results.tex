% https://www.overleaf.com/learn/latex/Inserting_Images
\section{Results}

Here we analyze the results to understand which dataset features most affect the power of a Cox proportional hazards model. We evaluate the value of the simulated datasets under our specifications, and build a model to estimate the power of the Cox proportional hazards model under our specifications. 

\subsection{Quality of Simulated Data}

We acknowledged three problems in choosing appropriate dataset parameters:

\begin{enumerate}
    \item How to choose dataset parameters which will generate datasets with values such that the Cox proportional hazards model converges?
    \item How to choose dataset parameters which are reasonably well distributed in the logit space? 
    \item How to choose dataset parameters to simulated datasets whose observed power is somewhat uniformly distributed between 0 and 1?
\end{enumerate}



We verify that chose appropriate dataset parameters by observing the distribution of observed power (Figure \ref{fig:hist}). The mode power is 0, however the values between 0 and 1 are reasonably uniform. Our decision to resample 5 observations from each level holds. 

\subsection{Logistic Models}

In this section we interpret the predicted effects of four models. We briefly review the model results before discussing the effects of dataset parameters separately. The estimates for each model are summarized in Tables \ref{tbl:lin-quad-models} and \ref{tbl:log-model} \cite{stargazer}. 

The linear-effects model (1) reinforces all our assumptions with no unexpected signs. Each dataset parameter has a large effect size and is significant at the $p < 0.01$ level. The quadratic-effects model (2) also fit as desired. Comparing their Akaike information criterion (AIC), the quadratic model appears to be a better specified model (their AICs are 2,534 and 2,180, respectively). 

The cubic-effects model is suspicious. First, many of the values are not significant. This may be due to high multicollinearity between first-, second-, and third-order arguments. 

We bring special attention to the fact that the effects of $\nu$ are not significant for the cubic- or diminishing-effects models, whereas they were significant for the linear- and quadratic-effects models. We will observe the effects directly to better understand this.

Finally, observe that the AIC of the four models descends in order, suggesting our diminishing-effects model is the most appropriate. Under further scrutiny, however, acknowledge that the quadratic- and cubic-effects models introduce two and three times as many parameters as the linear- or diminishing-effects models. The linear-effects model has the largest AIC in part because its a poor fit for the effects we model. 

\subsubsection{Baseline Hazard Rate $(\lambda)$}

Intuitively, there is a strictly inverse relationship between $\lambda$ and $\pi$; a study is more likely to find evidence for a true effect if the observation window is small. In accordance with this belief, every model in figure \ref{fig:power-baseline-hazard} suggests a lower $\lambda$ increases power. The effect is not as pronounced as we thought, however. For example, reducing the observation window by half, and therefore reducing $\lambda$ by half, adds little power to the study.

While every model agrees to a roughly linear relationship between $\lambda$ and $\pi$, the predicted power of each model must differ due to the differing effects on other variables. Let's continue.

\subsubsection{Treatment Hazard Ratio $(\exp[\beta])$}

Recall we defined values for $\beta$ such that $\beta < 1$. Therefore a larger $\beta$ implies a smaller treatment effect. A smaller treatment effect will, all else equal, be harder to identify. 

The relationship between $\beta$ and $\pi$ is inverse as shown in figure \ref{fig:power-treatment-hr}. The quadratic- and cubic-effects models appear incorrect at first, but a reasonable explanation exists. If the treatment all but guarantees immunity, then the treated individuals in our simulation would all be right-censored. This prompts a follow up question: Is the Cox proportional hazards model robust to unbalanced censored data? 

Notably, the third-order estimate of $\beta$ in the cubic-effects model is the smallest and is not significant (Table \ref{tbl:lin-quad-models}). The cubic-effects model is alone in this regard, as the other models fit large and significant estimates for $\beta$.

Furthermore, the same models indicate an increasingly strong inverse relationship for values of $\lambda \in [0.4, 1)$. Its likely the linear- and log-effects models are fitting on this trend as well. 

\subsubsection{Cohort Size $(s)$}

Figure \ref{fig:power-cohort-size} shows the estimated relationship of cohort size on power for each model. Of all the dataset parameters, this one seems best explained by the log-effects model. The marginal effect of the fifth patient is intuitively larger than the marginal effect of the 25th patient. Both the quadratic- and cubic-effects models show a second-order inversion which agrees with the diminishing effect assumption. 

A small study with four patients could greatly benefit from doubling their cohort size. 

\subsubsection{Expected Lifetimes $(\nu)$}

Expected lifetimes $(\nu)$ is the parameter indicating study-length. We're getting close the punch-line now. Again, we expect a diminishing effect and the models verify that. 

Power rapidly increases where $\nu < 2$ and slowly tapers off after $\nu = 2$. The effect is so pronounced that it seems reasonable to advise small studies observe at least 2 expected lifetimes before applying the Cox proportional hazards model --- when the constraint is possible. As figure (\ref{fig:power-expected-lifetimes-cohort-size}) shows, however, the effect is readily offset by the introduction of large cohorts. That is, short studies of several patients are more powerful than longer studies with few patients. 

\subsubsection{Share of Enrollment Periods Open $(\omega)$}

Finally, we arrive at the core question we set out to answer. How do we filter our dataset to maximize power?  Shockingly, not one model suggests an inversion point for valid values of $\omega$. This suggests that the marginal value of an additional 


\begin{itemize}
    \item It's always better to add more cohorts (Figure \ref{fig:power-pct-open-enrollment}).
\end{itemize}
